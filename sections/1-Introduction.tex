\begin{abstract}
A edição de superfícies suaves representadas por malhas de triângulos no computador é uma área ativa 
de pesquisa em Modelagem Geométrica, devido à sua gama cada vez maior de aplicações na indústria 
e no desenho artístico. Porém, a variedade de ferramentas de edição de superfícies disponíveis atualmente 
ainda é escassa. Problemas como deformação de forma livre, inserção de texturas geométricas, inpainting, 
reconstrução com restrições, remoção de ruído, reparos, entre outras, carecem de métodos robustos, 
flexíveis e eficientes que permitam edição interativa e que sejam capazes de reconstruir superfícies de 
qualidade. 
Neste trabalho, usamos a representação de superfícies baseada no operador gradiente discreto 
da malha, para desenvolver novas ferramentas interativas de edição de objetos 3d representados por malhas 
de triângulos.

% DO NOT USE SPECIAL CHARACTERS, SYMBOLS, OR MATH IN YOUR TITLE OR ABSTRACT.
%
\end{abstract}

\begin{IEEEkeywords}
one or two words; separated by semicolon; from specific; to generic fields;

\end{IEEEkeywords}


\IEEEpeerreviewmaketitle


% Wherever Times is specified, Times Roman or Times New Roman may be used. If neither is available on your system, please use the font closest in appearance to Times. Avoid using bit-mapped fonts if possible. True-Type 1 or Open Type fonts are preferred. Please embed symbol fonts, as well, for math, etc.

%==========================================
%==========================================


%==========================================
\section{Introdução}
%
Representação e processamento de objetos 3d é um dos mais importantes tópicos em Computação 
Gráfica e Modelagem Geométrica. A maneira como um objeto 3d é definido restringe o conjunto de 
operações que podem ser aplicadas a ele. Diferentes representações de objetos podem exibir diferentes 
propriedades geométricas, combinatórias e até perceptivas da forma. 
Uma das representações de superfícies mais pesquisadas nos últimos anos é a que se baseia no 
operador gradiente discreto em malhas triangulares. Esta representação fornece intrinsicamente as 
propriedades geométricas destas superfícies, e tem sido muito utilizada para resolver problemas de 
edição, onde destacamos [1].

No problema de edição de superfícies, deseja-se transformar uma superfície suave, representada no 
computador por uma malha de triângulos, em uma outra superfície, preservando características como 
continuidade, suavidade, detalhes, e ao mesmo tempo respeitando restrições impostas pelo usuário. A 
Figura 1 ilustra o problema.

Porém, a variedade de ferramentas de edição de superfícies disponíveis atualmente ainda é escassa. 
Problemas como deformação de forma livre, inserção de texturas geométricas, inpainting, reconstrução 
com restrições, remoção de ruído, reparos, entre outras, carecem de métodos robustos, flexíveis e 
eficientes que permitam edição interativa e que sejam capazes de reconstruir superfícies de qualidade. 
No projeto precedente, o operador Laplaciano e gradiente discreto foram estudados, e uma 
implementação computacional da equação de Poisson foi desenvolvida. Neste projeto, as propriedades do 
operador gradiente serão investigadas para o desenvolvimento de novas ferramentas de edição interativas 
de objetos 3d representados por malhas de triângulos. Estas ferramentas serão inspiradas na equação de 
Poisson, já adotada com sucesso para edição de imagens em [2].


%------------------------------------------------------------------------- 
\paragraph*{Contribuições}
%
[THALES ALTERE AQUI]


%------------------------------------------------------------------------- 
%\subsection{Related work}
%%
%We can roughly classify the approaches used for our application in three categories: first category, second category, and last category.\\
%Approaches in the first category were introduced by Pierre~\cite{Sibgrapi2014} using this and that techniques.
%
%
%\subimages[htb]{Technique overview}{overview}{%
%  \subimage[First step]{.31}{overviewa}%
%  \subimage[Second step]{.31}{overviewb}%
%  \subimage[Result]{.31}{overviewc}%
%}

