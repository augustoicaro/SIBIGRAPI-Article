\begin{abstract}
With the recent evolution of certain applications, there is a growing need for methods of this kind\ldots{}

This paper proposes exactly the right solution for this sub-problem in terms of several criterions. It introduces techniques for this and that tasks, improving this characteristic of the results. It further opens to a wider range of applications, as the experiments related in this paper confirms.

% DO NOT USE SPECIAL CHARACTERS, SYMBOLS, OR MATH IN YOUR TITLE OR ABSTRACT.
%
\end{abstract}

\begin{IEEEkeywords}
one or two words; separated by semicolon; from specific; to generic fields;

\end{IEEEkeywords}


\IEEEpeerreviewmaketitle


% Wherever Times is specified, Times Roman or Times New Roman may be used. If neither is available on your system, please use the font closest in appearance to Times. Avoid using bit-mapped fonts if possible. True-Type 1 or Open Type fonts are preferred. Please embed symbol fonts, as well, for math, etc.

%==========================================
%==========================================


%==========================================
\section{Introduction}
%
In the general context of this field, a certain kind of application has recently aggregated values for the following reasons.
%
However, existing approaches to produce good results for this application do not perform optimally yet, being limited to certain aspects and requiring too much resources to be actually used.


%------------------------------------------------------------------------- 
\paragraph*{Contributions}
%
This paper proposes a different approach to overcome those difficulties. By introducing and adapting those techniques to this context, we achieve significant improvements on the recent results. In particular, our method can handle this kind of data, and reduces the resource requirements. In our experiments, we evaluate a gain of $xx\%$ and could observe several interesting results that validate and delimit our approach.


%------------------------------------------------------------------------- 
\subsection{Related work}
%
We can roughly classify the approaches used for our application in three categories: first category, second category, and last category.\\
Approaches in the first category were introduced by Pierre~\cite{Sibgrapi2014} using this and that techniques.


\subimages[htb]{Technique overview}{overview}{%
  \subimage[First step]{.31}{overviewa}%
  \subimage[Second step]{.31}{overviewb}%
  \subimage[Result]{.31}{overviewc}%
}

