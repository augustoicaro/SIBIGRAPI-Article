%\subsection{Technique overview}
\subsection{Trabalhos Relacionados}

Para o desenvolvimento desse nosso trabalho, percebemos que métodos de edição de superfícies têm uma participação forte na área de Modelagem Geométrica. Podemos visualizar esse tipo de atividade em aplicações envolvendo indústrias ou manipulação de desenhos artísticos.

Durante muitos anos, a abordagem utilizada na literatura, envolvia uma representação baseada em superfícies paramétricas \cite{FarinG}, que podem ter uma generalização para domínios de base não regular usando técnicas de subdivisão \cite{SCHRZORI}.

Com o crescimento da divulgação dos Scanners 3D, a obtenção de geometrias de superfícies de objetos reais ficou acessível \cite{NEUGEBAUER}. Porém, esses novos tipos de superfícies apresentam uma amostragem densa e não suave, não sendo os métodos de edição de superfícies já existentes. Devido à isso, houve o surgimentos de novas ferramentas para edição desses tipos de superfícies .

Diversos métodos, surgiram até então para deformação de superfícies. Para o estado da arte dos métodos variacionais lineares, temos  \cite{Olga}, principalmente em relação métodos baseados no operador Laplaciano discreto \cite{Lang} e o Operador Gradiente Discreto \cite{Yu:2004}; que são versões adaptadas para superfícies diferenciadas, para malhas de triângulos.

Por fim, destacamos diversos problemas de edição que carecem de alternativas robustas e interativas. Ao nosso trabalho, a base de autovetores da matriz laplaciana de superfície é usada para realizar edição por meio de frequências.

\section{Visão Geral}

Sejam $\mathcal{S}$ e $\mathcal{D}$ superfícies discretas representada por malhas de triângulos. Por simplicidade, vamos considerar que ambas contêm apenas um componente conexo. Nosso objetivo é reconstruir uma nova superfície $\mathcal{S}'$, inserindo um detalhe geométrico representado por $\mathcal{D}$ em uma determinada região de $\mathcal{S}$, minimizando possíveis cicatrizes. 

Nosso método possui 3 etapas. Inicialmente, é realizado um alinhamento semi-automático, onde o usuário fornece correspondências desejadas entre pontos de $\mathcal{S}$ e $\mathcal{D}$. Em seguida, a conectividade de $\mathcal{S}'$ é obtida a partir de parametrizações conformes de $\mathcal{S}$ e $\mathcal{D}$, que são alinhadas e costuradas. Finalmente, a geometria de $\mathcal{S}'$ é reconstruída através de uma representação baseada em gradientes discretos.













%
%In order to produce this application, we start with this processing, followed by this technique. In order to cope with this challenge, we introduce this formulation to produce this intermediate result. The formulation leads to this type of system, which is efficiently solved by adapting this technique. The final result is produced by this transform. The whole process is schematized in 
%\figref{overview}.



%==========================================
%\section{Technical background}
%
%In this section, we detail this classical technique. The reader can find a more complete exposition in the work of Paul~\cite{Sibgrapi2014}.


%------------------------------------------------------------------------- 
%\subsection{Important concept}
%
%An \emph{important concept} is a type of object:
%\begin{definition}[Important concept]
%Given this and that, an object $X$ is an important concept if it respects the following properties\ldots{}
%\end{definition}


%------------------------------------------------------------------------- 
%\subsection{Usual adaptation}
%
%This concept has been used for applications similar to ours~\cite{Sibgrapi2014}, using the following formulation\ldots{}