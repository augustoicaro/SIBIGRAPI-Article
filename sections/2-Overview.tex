\subsection{Technique overview}
%
In order to produce this application, we start with this processing, followed by this technique. In order to cope with this challenge, we introduce this formulation to produce this intermediate result. The formulation leads to this type of system, which is efficiently solved by adapting this technique. The final result is produced by this transform. The whole process is schematized in \figref{overview}.



%==========================================
\section{Technical background}
%
In this section, we detail this classical technique. The reader can find a more complete exposition in the work of Paul~\cite{Sibgrapi2014}.


%------------------------------------------------------------------------- 
\subsection{Important concept}
%
An \emph{important concept} is a type of object:
\begin{definition}[Important concept]
Given this and that, an object $X$ is an important concept if it respects the following properties\ldots{}
\end{definition}


%------------------------------------------------------------------------- 
\subsection{Usual adaptation}
%
This concept has been used for applications similar to ours~\cite{Sibgrapi2014}, using the following formulation\ldots{}