\section{Reconstrução}
%
Após a seleção das correspondências entre a malha principal e a malha do detalhe a ser
adicionada e as parametrizações já estarem alinhadas, iremos trabalhar apenas com as parametrizaçãos
no $\mathbb{R}^2$. Definimos o conjunto de half-edges do bordo da malha detalhe como 
$B_d = {he_0, he_1, he_2, \ldots, he_n} | \forall n \in \mathbb{N}: O(he_n) = -1$ e
$B_m$ da mesma forma para a malha principal, onde $O( )$ retorna o oposto da half-edge
$he_n$, $-1$ se não existir e o indice do oposto caso exista, também definimos a poligonal
$P = {v_0, v_1, v_2, ..., v_n} | \forall n \in \mathbb{N}: v_n = vertex(he_n)$, onde $vertex( )$ retorna o vertice
da half-edge $he_n$. 

Sendo assim iremos excluir cada triângulo da parametrização da malha principal que intersecta ou é
interior à $P$. Com isso observamos a criação de um componente conexo como podemos ver na \figref{hole},
assim precisamos triangular o espaço entre as parametrizações. Para isto vamos usar uma triangulação com
restrições.

\subimages[htb]{Exclusão de triângulos pela parametrização}{hole}{
  \subimage{.48}{parametrization_hole1}%
  \subimage{.48}{parametrization_hole2}%
}

Em seguida iremos usar a Equação de Poisson para suavizar a colagem e ajustar melhor a malha ao espaço preenchido
pela nova triangulação reduzindo o tamanho dos novos triângulos e aumentando o tamanho dos triângulos do detalhe,
como podemos ver na \figref{reconstruction}.

%------------------------------------------------------------------------- 
\subsection{Triangulação com Restrições}
%
Usaremos a técnica de sweeping(varreduda) para realizar a triangulação, obedecendo o conjunto das
arestas da restrição demominado por $E = {e_0, e_1, e_2, \ldots, e_n} | \forall n \in \mathbb{N}: e_n$ é uma aresta
da restrição, onde os vértices das arestas da restrição  $ \subset (B_m \cup B_d)$, e o conjunto de arestas da
primeira vizinhança de $B_m$ e $B_d$ demominado por $E_v = {e_0, e_1, e_2, \ldots, e_m, e_{m+1}, e_{m+2}, \ldots
e_{m+n}} | \forall m, n \in \mathbb{N}: e_m$ é uma aresta da da primeira vizinhança de $B_m$ e $e_{M+n}$ é uma aresta
da primeira vizinhança de $B_d$, onde $M$ é o número de arestas da primeira vizinhança de $B_m$.

A técnica consistem em:
\begin{itemize}
 \item{Ordenar os vértices das arestas de $E_v$ de acordo com a coordena x;}
 \item{Ordenar os vértices de $B_m \cup B_d$ de acordo com a coordenada x;}
 \item{Testar a visibilidade de cada vértice com os vértices anteriores de ambos vetores;}
 \item{Se o ponto for visível, adicione a aresta à nova triangulação.}
\end{itemize}

Seguindo este algoritmo conseguimos uma triângulação que respeita as restrições, mas pode gerar triângulos muito
finos ou muito pequenos como podemos ver na \figref{triangulation}.

\subimages[htb]{Triângulação com restrições}{triangulation}{
  \subimage{.48}{parametrization_trianguled1}%
  \subimage{.48}{parametrization_trianguled2}%
}

%------------------------------------------------------------------------- 
\subsection{Equação de Poisson}

\subimages[htb]{Resolução do sistema com gradiente 0 para todos os triângulos gerados pela triângulação com restrição}{reconstruction}{
  \subimage{.48}{reconstruction1}%
  \subimage{.48}{reconstruction2}%
}