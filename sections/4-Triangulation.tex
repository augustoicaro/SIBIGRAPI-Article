\section{Reconstrução}
%
Após a seleção das correspondências entre a malha principal e a malha do detalhe a ser
adicionada e as parametrizações já estarem alinhadas, iremos trabalhar apendas com as parametrizaçãos
no $\mathbb{R}^2$. Definimos o conjunto de half-edges do bordo da malha detalhe como 
$B = {he_0, he_1, he_2, \ldots, he_n} | \forall n \in \mathbb{N}: O(he_n) = -1$, onde $O( )$ retorna o oposto da
half-edge $HE_n$, $-1$ se não existir e o indice do oposto caso exista, também definimos a poligonal
$P = {v_0, v_1, v_2, ..., v_n} | \forall n \in \mathbb{N}: v_n = vertex(HE_n)$, onde $vertex( )$ retorna o vertice
da half-edge. 

Sendo assim iremos excluir cada triângulo da parametrização da malha principal que intersecta ou é
interior à $P$. Com isso observamos a criação de um componente conexo como podemos ver na [figura 4.1],
assim precisamos triangular o espaço entre as parametrizações. Para isto vamos usar uma triangulação com
restrições.

%------------------------------------------------------------------------- 
\subsection{Triangulação com Restrições}
%
Usaremos a técnica de sweeping(varreduda) para realizar a triangulação, obedecendo o conjunto das
arestas da restrição $E = {e_0, e_1, e_2, \ldots, e_n} | \forall n \in \mathbb{N}: e_n$ é uma aresta da restrição,
ou seja, é uma aresta que deve fazer parte da nova triangulação.

%------------------------------------------------------------------------- 
\subsection{Equação de Poisson}